<+	+>	!comp!	!exe!
%% Based on <bare_jrnl.tex> in the ieee package available from CTAN,
%% I have changed the options so that most useful ones are clubbed together,
%% Have a look at <bare_jrnl.tex> to understand the function of each package. 

%% This code is offered as-is - no warranty - user assumes all risk.
%% Free to use, distribute and modify.

% *** Authors should verify (and, if needed, correct) their LaTeX system  ***
% *** with the testflow diagnostic prior to trusting their LaTeX platform ***
% *** with production work. IEEE's font choices can trigger bugs that do  ***
% *** not appear when using other class files.                            ***
% Testflow can be obtained at:
% http://www.ctan.org/tex-archive/macros/latex/contrib/supported/IEEEtran/testflow

%        File: !comp!expand("%:p:t")!comp!
%     Created: !comp!strftime("%a %b %d %I:00 %p %Y ").substitute(strftime('%Z'), '\<\(\w\)\(\w*\)\>\(\W\|$\)', '\1', 'g')!comp!
% Last Change: !comp!strftime("%a %b %d %I:00 %p %Y ").substitute(strftime('%Z'), '\<\(\w\)\(\w*\)\>\(\W\|$\)', '\1', 'g')!comp!
%
\documentclass[journal]{IEEEtran}

\usepackage{cite, graphicx, subfigure, amsmath} 
\interdisplaylinepenalty=2500

% *** Do not adjust lengths that control margins, column widths, etc. ***
% *** Do not use packages that alter fonts (such as pslatex).         ***
% There should be no need to do such things with IEEEtran.cls V1.6 and later.

<++>
% correct bad hyphenation here
\hyphenation{<+op-tical net-works semi-conduc-tor+>}


\begin{document}
%
% paper title
\title{<+Skeleton of IEEEtran.cls for Journals in VIM-Latex+>}
%
%
% author names and IEEE memberships
% note positions of commas and nonbreaking spaces ( ~ ) LaTeX will not break
% a structure at a ~ so this keeps an author's name from being broken across
% two lines.
% use \thanks{} to gain access to the first footnote area
% a separate \thanks must be used for each paragraph as LaTeX2e's \thanks
% was not built to handle multiple paragraphs
\author{<+Sumit Bhardwaj+>~\IEEEmembership{<+Student~Member,~IEEE,+>}
<+John~Doe+>,~\IEEEmembership{<+Fellow,~OSA,+>}
<+and~Jane~Doe,+>~\IEEEmembership{<+Life~Fellow,~IEEE+>}}% <-this % stops a space
\thanks{<+Manuscript received January 20, 2002; revised August 13, 2002.
This work was supported by the IEEE.+>}% <-this % stops a space
\thanks{<+S. Bhardwaj is with the Indian Institute of Technology, Delhi.+>}
%
% The paper headers
\markboth{<+Journal of VIM-\LaTeX\ Class Files,~Vol.~1, No.~8,~August~2002+>}{
<+Bhardwaj \MakeLowercase{\textit{et al.}+>}: <+Skeleton of IEEEtran.cls for Journals in VIM-Latex+>}
% The only time the second header will appear is for the odd numbered pages
% after the title page when using the twoside option.


% If you want to put a publisher's ID mark on the page
% (can leave text blank if you just want to see how the
% text height on the first page will be reduced by IEEE)
%\pubid{0000--0000/00\$00.00~\copyright~2002 IEEE}

% use only for invited papers
%\specialpapernotice{(Invited Paper)}

% make the title area
\maketitle


\begin{abstract}
<+The abstract goes here.+>
\end{abstract}

\begin{keywords}
<+IEEEtran, journal, \LaTeX, paper, template, VIM, VIM-\LaTeX+>.
\end{keywords}

\section{Introduction}
\PARstart{<+T+>}{<+his+>} <+demo file is intended to serve as a ``starter file"
for IEEE journal papers produced under \LaTeX\ using IEEEtran.cls version
1.6 and later.+>
% You must have at least 2 lines in the paragraph with the drop letter
% (should never be an issue)
<+May all your publication endeavors be successful.+>

% needed in second column of first page if using \pubid
%\pubidadjcol

% trigger a \newpage just before the given reference
% number - used to balance the columns on the last page
% adjust value as needed - may need to be readjusted if
% the document is modified later
%\IEEEtriggeratref{8}
% The "triggered" command can be changed if desired:
%\IEEEtriggercmd{\enlargethispage{-5in}}

% references section

%\bibliographystyle{IEEEtran.bst}
%\bibliography{IEEEabrv,../bib/paper}
\begin{thebibliography}{1}

\bibitem{IEEEhowto:kopka}
H.~Kopka and P.~W. Daly, \emph{A Guide to {\LaTeX}}, 3rd~ed.\hskip 1em plus
0.5em minus 0.4em\relax Harlow, England: Addison-Wesley, 1999.

\end{thebibliography}

% biography section
% 
\begin{biography}{Sumit Bhardwaj}
Biography text here.
\end{biography}

% if you will not have a photo
\begin{biographynophoto}{John Doe}
Biography text here.
\end{biographynophoto}

% insert where needed to balance the two columns on the last page
%\newpage

\begin{biographynophoto}{Jane Doe}
Biography text here.
\end{biographynophoto}

% You can push biographies down or up by placing
% a \vfill before or after them. The appropriate
% use of \vfill depends on what kind of text is
% on the last page and whether or not the columns
% are being equalized.

%\vfill

% Can be used to pull up biographies so that the bottom of the last one
% is flush with the other column.
%\enlargethispage{-5in}

\end{document}
